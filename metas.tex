\documentclass[a4paper]{article}

\usepackage[utf8]{inputenc}
\usepackage{hyperref}
\usepackage[portuguese]{babel}
\usepackage{ulem}

\pagenumbering{gobble}

\begin{document}

Descrição do trabalho:
\href{https://sites.google.com/site/christianersbrasil/classroom-news/descricaodotrabalho-gbc024cacb}
{LINK}.\\

Objetivos:
\begin{enumerate}
\item Ler um ou mais polinômios Pi(x);
\item \sout{Somar dois polinômios P1(x) e P2(x);}
\item \sout{Subtrair dois polinômios P1(x) e P2(x);}
\item Multiplicar dois polinômios P1(x) e P2(x); 
\item Dividir dois polinômios P1(x) e P2(x);
\item \sout{Calcular a derivada de um polinômio P(x);}
\item Simplificar um polinômio P(x);
\item \sout{Calcular o resultado de um determinado polinômio P(x) a partir de um valor de x,
 dado pelo o usuário (P(x) também é dado pelo usuário);}
\item Calcular o resultado de um determinado polinômio composto Q(P(x)), a partir de 
um valor de x, dado pelo o usuário, onde P(x) e Q(x) não são necessariamente 
diferentes (P(x) e Q(x) também são dados pelo usuário).
\item Calcular o resultado de um determinado polinômio composto por quantos 
polinômios o usuário desejar.
\item Salvar o log (histórico) de todos os polinômios, dados pelo usuário ou gerados 
por operações realizadas, até o momento em um arquivo texto (dos mais 
recentes para os mais antigos);
\item Sair.
\item Extra: \sout{Calcular a integral de um polinômio;}
\end{enumerate}

\end{document}
